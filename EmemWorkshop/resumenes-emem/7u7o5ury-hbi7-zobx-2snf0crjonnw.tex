\documentclass[12pt,twoside,twocolumn,english]{article}
\usepackage[T1]{fontenc}
\usepackage[latin9]{inputenc}
\usepackage{geometry}
\geometry{verbose,tmargin=3cm,bmargin=3cm,lmargin=2cm,rmargin=2cm}
\usepackage{fancyhdr}
\pagestyle{fancy}
\usepackage{babel}
\usepackage{url}
\usepackage{longtable}
\tolerance = 10000
\pretolerance = 10000
\setlength{\parindent}{0pt}
\usepackage[unicode=true,pdfusetitle,
 bookmarks=true,bookmarksnumbered=false,bookmarksopen=false,
 breaklinks=false,pdfborder={0 0 1},backref=false,colorlinks=false]
 {hyperref}

\makeatletter
%%%%%%%%%%%%%%%%%%%%%%%%%%%%%% User specified LaTeX commands.
\usepackage{fancyhdr}
\pagestyle{fancy}
\fancyhf{}

\fancyhead[RO]{5 Workshop de Educaci�n Matem�tica, Estad�stica y Matem�ticas - EMEM 2019, 5-7 de noviembre de 2019, Armenia, Colombia.}

\fancyfoot[LO]{  \rule[0.25ex]{1\columnwidth}{1pt} \\ Programa de Licenciatura en Matem�ticas, Universidad del Quind�o \hfill{} \thepage}

\setlength\columnsep{1cm}

\makeatother

\begin{document}
\title{Generaci�n autom�tica de mallas para discretizar dominios bidimensionales}
\author{Estiven\thanks{Universidad del Quindio, eflorezc_1@uqvirtual.edu.co}  }
\date{Noviembre del 2019}
\maketitle
\thispagestyle{fancy}
\subsection*{Palabras Clave}
Mallas
\rule[0.25ex]{1\columnwidth}{1pt}
Resumen:
Este p�ster presentara un m�todo para la discretizaci�n de dominios bidimensionales mediante el c�digo SpecMesh2d basado en el lenguaje fortran, para ello es necesario el estudio de que es una malla estructurada y no-estructurada, conocer algunos de los m�todos en lo que se puede generar una malla dependiendo el dominio que se necesite discretizar. Se conocer� un poco de lo que puede ofrecer el c�digo SpecMesh2D en gfortran, para el cual se necesitara un visualizador para las gr�ficas, que se deben programar en alg�n editor de c�digo dadas ecuaciones param�tricas, dado esto se creara un archivo el cual visualizaremos en los resultados del p�ster tanto las caracter�sticas que llevara la discretizaci�n como la el dominio que se va a discretar.

I
\begin{thebibliography}{1}
\bibitem{ref1}Calder�n,Elkinn., \& A. Minoli, C�sar A. (2015). \textsl{Generador autom�tico de mallas cuadrangulares para la soluci�n num�rica de problemas de propagaci�n de ondas}. 
\end{thebibliography}
\end{document}