\documentclass[12pt,twoside,twocolumn,english]{article}
\usepackage[T1]{fontenc}
\usepackage[latin9]{inputenc}
\usepackage{geometry}
\geometry{verbose,tmargin=3cm,bmargin=3cm,lmargin=2cm,rmargin=2cm}
\usepackage{fancyhdr}
\pagestyle{fancy}
\usepackage{babel}
\usepackage{url}
\usepackage{longtable}
\tolerance = 10000
\pretolerance = 10000
\setlength{\parindent}{0pt}
\usepackage[unicode=true,pdfusetitle,
 bookmarks=true,bookmarksnumbered=false,bookmarksopen=false,
 breaklinks=false,pdfborder={0 0 1},backref=false,colorlinks=false]
 {hyperref}

\makeatletter
%%%%%%%%%%%%%%%%%%%%%%%%%%%%%% User specified LaTeX commands.
\usepackage{fancyhdr}
\pagestyle{fancy}
\fancyhf{}

\fancyhead[RO]{5 Workshop de Educaci�n Matem�tica, Estad�stica y Matem�ticas - EMEM 2019, 5-7 de noviembre de 2019, Armenia, Colombia.}

\fancyfoot[LO]{  \rule[0.25ex]{1\columnwidth}{1pt} \\ Programa de Licenciatura en Matem�ticas, Universidad del Quind�o \hfill{} \thepage}

\setlength\columnsep{1cm}

\makeatother

\begin{document}
\title{Evaluaci�n del consumo m�ximo de oxigeno de las selecciones deportivas de la universidad del Quind�o a partir del test de Leger utilizando una herramienta matem�tica-computacional}
\author{C�sar A. Zambrano Benjumea\thanks{Universidad del Quind�o, Cazambranob@uqvirtual.edu.co}  }
\date{Noviembre del 2019}
\maketitle
\thispagestyle{fancy}
\subsection*{Palabras Clave}
Test de Leger
\rule[0.25ex]{1\columnwidth}{1pt}
El p�ster consta de como por medio de una herramienta matem�tica: ajustes lineales por m�nimos cuadrados, nos permite obtener unos resultados mas claros y concisos cuando se esta evaluando la capacidad f�sica de la resistencia por medio del test de leger o course-navette, este test fundamentalmente permite medir el Vo2max de una persona, es decir, permite medir el consumo m�ximo de oxigeno de una persona en determinado tiempo, teniendo en cuenta factores como su edad, la velocidad con la que realiza el test, entre otras.
El test es algo muy sencillo, se ubican dos conos a veinte (20) metros de distancia entre ellos, la persona o grupo de personas; en este caso deportistas de las selecciones de la universidad del Quind�o;  tiene una se�al sonora que le da el momento de salida en cada cono, si llegan al cono despu�s de haberse emitido la se�al sonora tendr� que salir de la prueba inmediatamente. A medida que el tiempo transcurre la se�al sonora se emitir� cada vez mas r�pido obligando a los participantes a acelerar su ritmo para llegar al cono antes de que la se�al sea emitida.
\begin{thebibliography}{1}
\bibitem{ref1}Kayihan, G., �zkan, A., K�kl�, Y., Eyuboglu, E., Ak�a, F., Koz, M., & Ers�z, G. (2014). Comparative analysis of the 1-mile run test evaluation formulae: Assessment of aerobic capacity in male law enforcement officers aged 20-23 years. International Journal of Occupational Medicine and Environmental Health, 27(2)
\bibitem{ref2}L�ger, L.A. & Lambert, J. Europ. J. Appl. Physiol. (1982) 49: 1. https://doi.org/10.1007/BF00428958
\bibitem{ref3}Molina Sotomayor, E., & Arcay Montoya, R. (19 de 05 de 2005). Portal Fitness. Obtenido de http://www.portalfitness.com/392_test-de-campo-naveta-con-periodos-de-1-minuto-para-estimar-el-vo2max.aspx
\end{thebibliography}
\end{document}