\documentclass[12pt,twoside,twocolumn,english]{article}
\usepackage[T1]{fontenc}
\usepackage[latin9]{inputenc}
\usepackage{geometry}
\geometry{verbose,tmargin=3cm,bmargin=3cm,lmargin=2cm,rmargin=2cm}
\usepackage{fancyhdr}
\pagestyle{fancy}
\usepackage{babel}
\usepackage{url}
\usepackage{longtable}
\tolerance = 10000
\pretolerance = 10000
\setlength{\parindent}{0pt}
\usepackage[unicode=true,pdfusetitle,
 bookmarks=true,bookmarksnumbered=false,bookmarksopen=false,
 breaklinks=false,pdfborder={0 0 1},backref=false,colorlinks=false]
 {hyperref}

\makeatletter
%%%%%%%%%%%%%%%%%%%%%%%%%%%%%% User specified LaTeX commands.
\usepackage{fancyhdr}
\pagestyle{fancy}
\fancyhf{}

\fancyhead[RO]{5 Workshop de Educaci�n Matem�tica, Estad�stica y Matem�ticas - EMEM 2019, 5-7 de noviembre de 2019, Armenia, Colombia.}

\fancyfoot[LO]{  \rule[0.25ex]{1\columnwidth}{1pt} \\ Programa de Licenciatura en Matem�ticas, Universidad del Quind�o \hfill{} \thepage}

\setlength\columnsep{1cm}

\makeatother

\begin{document}
\title{Aprendizaje del concepto de funci�n cuadr�tica en estudiantes de grado noveno, mediante la teor�a de las situaciones did�cticas  de Brousseau}
\author{}
\date{Noviembre del 2019}
\maketitle
\thispagestyle{fancy}
\subsection*{Palabras Clave}
Brousseau
\rule[0.25ex]{1\columnwidth}{1pt}
Resumen

 

      La presente investigaci�n permitir� conocer m�s de cerca las dificultades que presentan los estudiantes de grado noveno, en el aprendizaje de las matem�ticas y las funciones cuadr�ticas y establecer propuestas que permitan mejorar la comprensi�n del concepto de funci�n cuadr�tica.

 

El trabajo se realiza en el Colegio Jorge Isaac de la Ciudad de Armenia tomando como objeto un grupo de matem�ticas de grado noveno que presento dificultades las pruebas  de estado y en un diagn�stico previo se evidencio problemas en reconocer la ra�ces, simetr�a, v�rtice, concavidad en una funci�n, y no pose�an las herramientas necesarias en algebra b�sica, tales como el procedimiento de  despeje de variables o soluci�n de ecuaciones, tambi�n presentaban mucha dificultad sobre la interpretaci�n y manejo de la Ley de los signos en: suma, multiplicaci�n, divisi�n, potencias

 

Se propuso una estrategia para el aprendizaje de la funci�n cuadr�tica con la teor�a de las situaciones did�cticas de Brousseau, para mejorar la comprensi�n de dicho concepto. La estrategia dise�ada se implementa mediante la llevada a la pr�ctica de varias situaciones en el marco de la aplicaci�n de la ingenier�a did�ctica, la cual permite la validaci�n de la propuesta. 
\begin{thebibliography}{1}
\bibitem{ref1}Roa Africano, N. J. (2018). La funci�n cuadr�tica desde los sistemas de representaci�n simb�lico y gr�fico. 13. Bogota D.C., Colombia.
\end{thebibliography}
\end{document}