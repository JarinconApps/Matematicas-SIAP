\documentclass[12pt,twoside,twocolumn,english]{article}
\usepackage[T1]{fontenc}
\usepackage[latin9]{inputenc}
\usepackage{geometry}
\geometry{verbose,tmargin=3cm,bmargin=3cm,lmargin=2cm,rmargin=2cm}
\usepackage{fancyhdr}
\pagestyle{fancy}
\usepackage{babel}
\usepackage{url}
\usepackage{longtable}
\tolerance = 10000
\pretolerance = 10000
\setlength{\parindent}{0pt}
\usepackage[unicode=true,pdfusetitle,
 bookmarks=true,bookmarksnumbered=false,bookmarksopen=false,
 breaklinks=false,pdfborder={0 0 1},backref=false,colorlinks=false]
 {hyperref}

\makeatletter
%%%%%%%%%%%%%%%%%%%%%%%%%%%%%% User specified LaTeX commands.
\usepackage{fancyhdr}
\pagestyle{fancy}
\fancyhf{}

\fancyhead[RO]{5 Workshop de Educaci�n Matem�tica, Estad�stica y Matem�ticas - EMEM 2019, 5-7 de noviembre de 2019, Armenia, Colombia.}

\fancyfoot[LO]{  \rule[0.25ex]{1\columnwidth}{1pt} \\ Programa de Licenciatura en Matem�ticas, Universidad del Quind�o \hfill{} \thepage}

\setlength\columnsep{1cm}

\makeatother

\begin{document}
\title{Ajuste por Medio de Splines Bidimensionales para un Conjunto de Datos de Solidos Suspendidos Totales (SST)}
\author{Jordy Felipe Diaz Bermeo \thanks{Universidad del quindio , jfdiazb@uqvirtual.edu.co}  }
\date{Noviembre del 2019}
\maketitle
\thispagestyle{fancy}
\subsection*{Palabras Clave}
S�lidos suspendidos totales, Interpolac�n, Splines c�bicos, Splines bidimensionales, Python
\rule[0.25ex]{1\columnwidth}{1pt}
Est� trabajo consiste en determinar la superficie asociada a un conjunto de datos de s�lidos suspendidos totales (SST) en los meses octubre y noviembre del a�o 2018. En primer lugar, se someten estos datos al m�todo de splines c�bicos en Python [1] para generar una curva suave que pueda predecir con m�s precisi�n
el comportamiento de los datos y finalmente se construye la superficie utilizando splines bidimensionales [2].
\begin{thebibliography}{1}
\bibitem{ref1} Cerquera, Y. (2007). Interpolaci�n con trazadores o splines.
\bibitem{ref2}Baltazar, J. M., & E�a, L. R. (2006). Generaci�n de Mallas Estructuradas en Superficie. Informaci�n tecnol�gica, 17(3), 107-116.
\end{thebibliography}
\end{document}