\documentclass[12pt,twoside,twocolumn,english]{article}
\usepackage[T1]{fontenc}
\usepackage[latin9]{inputenc}
\usepackage{geometry}
\geometry{verbose,tmargin=3cm,bmargin=3cm,lmargin=2cm,rmargin=2cm}
\usepackage{fancyhdr}
\pagestyle{fancy}
\usepackage{babel}
\usepackage{url}
\usepackage{longtable}
\tolerance = 10000
\pretolerance = 10000
\setlength{\parindent}{0pt}
\usepackage[unicode=true,pdfusetitle,
 bookmarks=true,bookmarksnumbered=false,bookmarksopen=false,
 breaklinks=false,pdfborder={0 0 1},backref=false,colorlinks=false]
 {hyperref}

\makeatletter
%%%%%%%%%%%%%%%%%%%%%%%%%%%%%% User specified LaTeX commands.
\usepackage{fancyhdr}
\pagestyle{fancy}
\fancyhf{}

\fancyhead[RO]{5 Workshop de Educaci�n Matem�tica, Estad�stica y Matem�ticas - EMEM 2019, 5-7 de noviembre de 2019, Armenia, Colombia.}

\fancyfoot[LO]{  \rule[0.25ex]{1\columnwidth}{1pt} \\ Programa de Licenciatura en Matem�ticas, Universidad del Quind�o \hfill{} \thepage}

\setlength\columnsep{1cm}

\makeatother

\begin{document}
\title{Modelo hospedero parasitoide entre la abeja de especie Apis mellifera y el �caro Varroa destructor considerando tiempo de retardo distribuido}
\author{Carlos Andr�s Trujillo Salazar\thanks{Universidad del Quind�o, catrujillo@uniquindio.edu.co   }  , Yuliana Rodr�guez Badillo\thanks{Universidad del Quind�o, yrodriguezb_1@uqvirtual.edu.co}  }
\date{Noviembre del 2019}
\maketitle
\thispagestyle{fancy}
\subsection*{Palabras Clave}
Varroa destructor, hospedero parasitoide, Apis mellifera, Umbral ecol�gico, tiempos de retardo
\rule[0.25ex]{1\columnwidth}{1pt}

Las abejas desempe�an un papel importante en el sector agr�cola, econ�mico y medio ambiental. Su acci�n como agente polonizador permite la reproducci�n de especies de plantas florales y de otros cultivos nutricionales. De hecho, algunos autores sostienen que la alimentaci�n humana depende directamente de la funci�n de estos insectos. Desafortunadamente el n�mero de abejas en el mundo se ha reducido dr�sticamente debido a la acci�n del hombre y tambi�n de enemigos naturales, entre los que se destaca el �caro Varroa destructor. Teniendo en cuenta lo anterior, se propone el an�lisis de un modelo matem�tico del tipo hospedero-parasitoide, entre las especies abeja Apis mellifera y el par�sito �caro Varroa destructor. El modelo est� basado en ecuaciones diferenciales ordinarias no lineales y luego se incorpora un tiempo de retardo distribuido, lo que da lugar a un nuevo sistema integro- diferencial, para el que se hace an�lisis cualitativo, simulaciones num�ricas y se establecen condiciones de coexistencia.
\begin{thebibliography}{1}
\bibitem{ref1}Farkas, M. (2013). Periodic motions (Vol. 104). Springer Science & Business Media.
\bibitem{ref2}Herrero, F. (2004). Lo que usted debe saber sobre las abejas y la miel. Edici�n Caja Espa�a. Imprime Rub�n, SL.
\bibitem{ref3}Iggidr, A., Mbang, J., & Sallet, G. (2007). Stability analysis of within-host parasite models with delays. Mathematical biosciences, 209(1), 51-75.
\bibitem{ref4}Quero, A. (2004). Las abejas y la apicultura. Universidad de Oviedo Vicerrectorado de Extensi�n Universitaria, 48.
\end{thebibliography}
\end{document}