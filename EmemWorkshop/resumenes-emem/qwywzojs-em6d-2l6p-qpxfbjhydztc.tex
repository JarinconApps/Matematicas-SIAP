\documentclass[12pt,twoside,twocolumn,english]{article}
\usepackage[T1]{fontenc}
\usepackage[latin9]{inputenc}
\usepackage{geometry}
\geometry{verbose,tmargin=3cm,bmargin=3cm,lmargin=2cm,rmargin=2cm}
\usepackage{fancyhdr}
\pagestyle{fancy}
\usepackage{babel}
\usepackage{url}
\usepackage{longtable}
\tolerance = 10000
\pretolerance = 10000
\setlength{\parindent}{0pt}
\usepackage[unicode=true,pdfusetitle,
 bookmarks=true,bookmarksnumbered=false,bookmarksopen=false,
 breaklinks=false,pdfborder={0 0 1},backref=false,colorlinks=false]
 {hyperref}

\makeatletter
%%%%%%%%%%%%%%%%%%%%%%%%%%%%%% User specified LaTeX commands.
\usepackage{fancyhdr}
\pagestyle{fancy}
\fancyhf{}

\fancyhead[RO]{5 Workshop de Educaci�n Matem�tica, Estad�stica y Matem�ticas - EMEM 2019, 5-7 de noviembre de 2019, Armenia, Colombia.}

\fancyfoot[LO]{  \rule[0.25ex]{1\columnwidth}{1pt} \\ Programa de Licenciatura en Matem�ticas, Universidad del Quind�o \hfill{} \thepage}

\setlength\columnsep{1cm}

\makeatother

\begin{document}
\title{An�lisis de las dificultades que enfrentan los estudiantes en la resoluci�n de problemas, que se modelan a partir de una ecuaci�n de la forma ax+b=cx+d}
\author{Luis Fernando Andrade Fl�rez\thanks{Universidad del Quind�o, lfandradef@uqvirtual.edu.co}  }
\date{Noviembre del 2019}
\maketitle
\thispagestyle{fancy}
\subsection*{Palabras Clave}
Resoluci�n de Problemas , Ecuaciones Lineales
\rule[0.25ex]{1\columnwidth}{1pt}
La ense�a de ecuaciones de primer grado en la resoluci�n de problemas ha sido un tema de mucha dificultad para los estudiantes, puesto que les cuesta mucho inconveniente poder interpretar la lectura del problema y transcribirla en la expresi�n algebraica. 
Uno de los mayores problemas que se enfrenta la educaci�n actual es la falta de aplicabilidad de los conceptos que se manejan en el aula de clases, sin embargo es cierto que ejemplarizar todo concepto matem�tico muchas veces no es sencillo, pero si es necesario si se requiere un aprendizaje significativo.
En el proceso de ense�anza-aprendizaje de las matem�ticas, muchos autores han trabajado en el hecho de resolver problemas y no una gran miscel�nea de ejercicios, facilita el aprendizaje debido a las acciones t�picas del pensamiento que intervienen en la soluci�n de este, pues implica la intervenci�n de otros procesos de pensamiento tales como: la b�squeda de conexiones, el empleo de distintas representaciones, la necesidad de justificar los pasos dados en la soluci�n de un problema y comunicar los resultados obtenidos.
Finalmente, el prop�sito de esta investigaci�n es analizar mediante los recursos como: entrevistas escritas y verbales, registro filmacion y el registro del software educativo balanza de �ngel, las dificultades que presentan los estudiantes en la resoluci�n de problemas en el tema de ecuaciones de primer grado. En consecuencia, se realizar� una secuencia did�ctica adecuada para la aplicaci�n previa en los discentes; con el fin de extraer datos relevantes en cuanto a las dificultades que enfrentan los estudiantes al resolver dichos problemas. Teniendo en cuenta las evidencias obtenidas y por ende esto permitir� efectuar el respectivo an�lisis de las mismas, para determinar las conclusiones objeto de la presente investigaci�n.

\begin{thebibliography}{1}
\end{thebibliography}
\end{document}