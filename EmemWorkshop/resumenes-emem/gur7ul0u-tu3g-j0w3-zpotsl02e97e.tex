\documentclass[12pt,twoside,twocolumn,english]{article}
\usepackage[T1]{fontenc}
\usepackage[latin9]{inputenc}
\usepackage{geometry}
\geometry{verbose,tmargin=3cm,bmargin=3cm,lmargin=2cm,rmargin=2cm}
\usepackage{fancyhdr}
\pagestyle{fancy}
\usepackage{babel}
\usepackage{url}
\usepackage{longtable}
\tolerance = 10000
\pretolerance = 10000
\setlength{\parindent}{0pt}
\usepackage[unicode=true,pdfusetitle,
 bookmarks=true,bookmarksnumbered=false,bookmarksopen=false,
 breaklinks=false,pdfborder={0 0 1},backref=false,colorlinks=false]
 {hyperref}

\makeatletter
%%%%%%%%%%%%%%%%%%%%%%%%%%%%%% User specified LaTeX commands.
\usepackage{fancyhdr}
\pagestyle{fancy}
\fancyhf{}

\fancyhead[RO]{5 Workshop de Educaci�n Matem�tica, Estad�stica y Matem�ticas - EMEM 2019, 5-7 de noviembre de 2019, Armenia, Colombia.}

\fancyfoot[LO]{  \rule[0.25ex]{1\columnwidth}{1pt} \\ Programa de Licenciatura en Matem�ticas, Universidad del Quind�o \hfill{} \thepage}

\setlength\columnsep{1cm}

\makeatother

\begin{document}
\title{Estimaci�n de par�metros para un modelo matem�tico basado en ecuaciones diferenciales ordinarias que describe la din�mica poblacional de la broca del caf� (Hypothenemus hampei)}
\author{Juan Sebastian Marmolejo G�mez \thanks{Universidad del Qu�ndio, sebasmarmolejo@gmail.com}  }
\date{Noviembre del 2019}
\maketitle
\thispagestyle{fancy}
\subsection*{Palabras Clave}

\rule[0.25ex]{1\columnwidth}{1pt}
La broca del caf� (Hypothenemus hampei), es sido considerada la plaga de mayor importancia en el cultivo del caf� en el �mbito mundial, ocasionando graves da�os en los granos por causa de la perforaci�n generando la ca�da prematura de los frutos. Colombia, por ser una regi�n que favorece al crecimiento e infestaci�n de esta plaga debido a factores como el clima, ubicaci�n geogr�fica, producci�n continua de caf�, y dado que la caficultura es una de las actividades agr�colas m�s importantes en el pa�s, es importante realizar estudios referentes a la din�mica poblacional de la broca. Es por ello, que este trabajo tiene como objetivo analizar un modelo matem�tico basado en ecuaciones diferenciales ordinarias que describa la din�mica poblacional de la broca del caf� (Hypothenemus hampei). El modelo es tomado de la literatura, pero es estudiado solo num�ricamente, por lo que se lleva a cabo an�lisis cualitativo.

\begin{thebibliography}{1}
\end{thebibliography}
\end{document}