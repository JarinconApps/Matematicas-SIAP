\documentclass[12pt,twoside,twocolumn,english]{article}
\usepackage[T1]{fontenc}
\usepackage[latin9]{inputenc}
\usepackage{geometry}
\geometry{verbose,tmargin=3cm,bmargin=3cm,lmargin=2cm,rmargin=2cm}
\usepackage{fancyhdr}
\pagestyle{fancy}
\usepackage{babel}
\usepackage{url}
\usepackage{longtable}
\tolerance = 10000
\pretolerance = 10000
\setlength{\parindent}{0pt}
\usepackage[unicode=true,pdfusetitle,
 bookmarks=true,bookmarksnumbered=false,bookmarksopen=false,
 breaklinks=false,pdfborder={0 0 1},backref=false,colorlinks=false]
 {hyperref}

\makeatletter
%%%%%%%%%%%%%%%%%%%%%%%%%%%%%% User specified LaTeX commands.
\usepackage{fancyhdr}
\pagestyle{fancy}
\fancyhf{}

\fancyhead[RO]{5 Workshop de Educaci�n Matem�tica, Estad�stica y Matem�ticas - EMEM 2019, 5-7 de noviembre de 2019, Armenia, Colombia.}

\fancyfoot[LO]{  \rule[0.25ex]{1\columnwidth}{1pt} \\ Programa de Licenciatura en Matem�ticas, Universidad del Quind�o \hfill{} \thepage}

\setlength\columnsep{1cm}

\makeatother

\begin{document}
\title{Modelo de control biol�gico aplicado a la etapa adulta del mosquito Ae. aegypti}
\author{Carlos Alberto Abello Mu�oz\thanks{Universidad del Quind�o, carlosalbert15@gmail.com}  , Juan Felipe Ciro Sol�rzano\thanks{Universidad del Quind�o, jfciros@uqvirtual.edu.co}  }
\date{Noviembre del 2019}
\maketitle
\thispagestyle{fancy}
\subsection*{Palabras Clave}
Aedes aegypti, Depredaci�n, Control biol�gico, Crecimiento log�stico, Modelo matem�tico.
\rule[0.25ex]{1\columnwidth}{1pt}
El mosquito Ae. aegypti es el principal transmisor de virus como el dengue, el chikungunya y el zika. Habita principalmente en zonas tropicales y subtropicales que permiten su reproducci�n y propagaci�n convirti�ndose en una problem�tica de salud p�blica, dado que cerca de dos tercios de la poblaci�n mundial conviven con este vector en �reas urbanas. En la actualidad existen varios estudios que buscan encontrar la manera m�s �ptima de mitigar dicha problem�tica, siendo una de ellas el control biol�gico presa-depredador, que consiste en la introducci�n de especies aut�ctonas que logren controlar o reducir la plaga de alg�n modo. En este estudio se plantea un modelo presa-depredador donde el mosquito presenta un crecimiento constante y su depredador un crecimiento log�stico siendo codependientes el uno del otro, pero dejando al depredador de manera impl�cita (Vease como una especie que se alimenta de mosquitos adultos). De este estudio se espera proponer un modelo de control biol�gico tipo presa-depredador que busca determinar la viabilidad de los depredadores de Ae. aegypti en su etapa adulta, para dise�ar una alternativa en la reducci�n de las tasas de infecci�n de enfermedades transmitidas por el vector.
\begin{thebibliography}{1}
\bibitem{ref1}Organizaci�n mundial de la salud, Control Biol�gico. https://www.who.int/denguecontrol/control-strategies/biological-control/es
\bibitem{ref2}Pinheiro FP. Dengue in the Americas. Epidemiological Bulletin panamerican health organization. ISSN 0256-1859. Vol 10. No.1 1989. Epidemiol. Bull. PAHO. 1989. 1-8. http://hist.library.paho.org/English/EPID/8345.pdf
\bibitem{ref3}Rivera Garc�a, Oscar \textit{Aedes aegypti}, virus dengue, chinkugunia, zika y el cambio clim�tico. M�xima alerta m�dica y oficial REDVET. Revista Electr�nica de Veterinaria, vol. 15, n�m. 10, octubre, 2014, pp. 1-10 http://www.redalyc.org/pdf/636/63637999001.pdf
\end{thebibliography}
\end{document}