\documentclass[12pt,twoside,twocolumn,english]{article}
\usepackage[T1]{fontenc}
\usepackage[latin9]{inputenc}
\usepackage{geometry}
\geometry{verbose,tmargin=3cm,bmargin=3cm,lmargin=2cm,rmargin=2cm}
\usepackage{fancyhdr}
\pagestyle{fancy}
\usepackage{babel}
\usepackage{url}
\usepackage{longtable}
\tolerance = 10000
\pretolerance = 10000
\setlength{\parindent}{0pt}
\usepackage[unicode=true,pdfusetitle,
 bookmarks=true,bookmarksnumbered=false,bookmarksopen=false,
 breaklinks=false,pdfborder={0 0 1},backref=false,colorlinks=false]
 {hyperref}

\makeatletter
%%%%%%%%%%%%%%%%%%%%%%%%%%%%%% User specified LaTeX commands.
\usepackage{fancyhdr}
\pagestyle{fancy}
\fancyhf{}

\fancyhead[RO]{5 Workshop de Educaci�n Matem�tica, Estad�stica y Matem�ticas - EMEM 2019, 5-7 de noviembre de 2019, Armenia, Colombia.}

\fancyfoot[LO]{  \rule[0.25ex]{1\columnwidth}{1pt} \\ Programa de Licenciatura en Matem�ticas, Universidad del Quind�o \hfill{} \thepage}

\setlength\columnsep{1cm}

\makeatother

\begin{document}
\title{Mejoramiento del proceso de ense�anza - aprendizaje de la graficaci�n de funciones trigonom�tricas}
\author{Alejandra Giraldo Sanabria\thanks{Universidad del Quindio, agiraldos@uqvirtual.edu.co}  , Yarleni Perez Rojas\thanks{Universidad del Quindio, yperezr@uqvirtual.edu.co}   \\ Juli�n Andr�s Rinc�n Penagos\thanks{Universidad del Quind�o, jarincon@uniquindio.edu.co}  }
\date{Noviembre del 2019}
\maketitle
\thispagestyle{fancy}
\subsection*{Palabras Clave}
Trigonom�tria
\rule[0.25ex]{1\columnwidth}{1pt}
En este trabajo se implementara una estrategia metodol�gica para mejorar los niveles de aprendizaje en los estudiantes cuando se ense�an  las funciones trigonom�tricas , ya que las dificultades que se pueden ver es que no tienen en cuenta al momento de  hacer la gr�fica , como encontrar el periodo e  identificar la amplitud, el desplazamiento , son los errores m�s comunes que presentan los estudiantes cu�ndo deben graficar funciones .Por lo tanto se har�  uso de un software educativo y herramientas apropiadas que permita un buen trabajo con la graficaci�n de funciones trigonom�tricas articulando los conceptos vistos en clase .
\begin{thebibliography}{1}
\bibitem{ref1}VILCHEZ, J. (2005).LA ENSE�ANZA DE LAS FUNCIONES TRIGONOM�TRICAS EN EL QUINTO GRADO DE EDUCACI�N  SECUNDARIA . (Tesis de Maestr�a). PONTIFICIA UNIVERSIDAD CAT�LICA DEL PER�: lIMA 
\end{thebibliography}
\end{document}