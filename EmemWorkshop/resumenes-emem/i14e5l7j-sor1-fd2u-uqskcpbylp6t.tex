\documentclass[12pt,twoside,twocolumn,english]{article}
\usepackage[T1]{fontenc}
\usepackage[latin9]{inputenc}
\usepackage{geometry}
\geometry{verbose,tmargin=3cm,bmargin=3cm,lmargin=2cm,rmargin=2cm}
\usepackage{fancyhdr}
\pagestyle{fancy}
\usepackage{babel}
\usepackage{url}
\usepackage{longtable}
\tolerance = 10000
\pretolerance = 10000
\setlength{\parindent}{0pt}
\usepackage[unicode=true,pdfusetitle,
 bookmarks=true,bookmarksnumbered=false,bookmarksopen=false,
 breaklinks=false,pdfborder={0 0 1},backref=false,colorlinks=false]
 {hyperref}

\makeatletter
%%%%%%%%%%%%%%%%%%%%%%%%%%%%%% User specified LaTeX commands.
\usepackage{fancyhdr}
\pagestyle{fancy}
\fancyhf{}

\fancyhead[RO]{5 Workshop de Educaci�n Matem�tica, Estad�stica y Matem�ticas - EMEM 2019, 5-7 de noviembre de 2019, Armenia, Colombia.}

\fancyfoot[LO]{  \rule[0.25ex]{1\columnwidth}{1pt} \\ Programa de Licenciatura en Matem�ticas, Universidad del Quind�o \hfill{} \thepage}

\setlength\columnsep{1cm}

\makeatother

\begin{document}
\title{Uso de elementos del laboratorio de matem�ticas en el aula de clase}
\author{Francy Pati�o Jim�nez \thanks{Rom�n Mar�a Valencia de Calarc�, panchy6501@gmail.com}  , Martha Cecilia Ram�rez Rodriguez\thanks{Rom�n Mar�a Valencia de Calarc�, marthace87@gmail.com}   \\ Jhoan Esteban Soler Giraldo\thanks{Rom�n Mar�a Valencia de Calarc�, jhoan011989@gmail.com}  , Juan Jos� Bernal\thanks{Rom�n Maria Valencia de Calarc�, Jjbe2004@gmail.com}  , Nayru Alexandra Ram�rez Molano\thanks{Rom�n Mar�a Valencia, nanualexandraramirez@gmail.com}  }
\date{Noviembre del 2019}
\maketitle
\thispagestyle{fancy}
\subsection*{Palabras Clave}
Laboratorio de Matem�ticas, Material Did�ctico
\rule[0.25ex]{1\columnwidth}{1pt}
Los materiales del laboratorio de matem�ticas, son una herramienta did�ctica que apoyan el aprendizaje, ayudando a pensar, incitando la imaginaci�n y creaci�n, ejercitando la manipulaci�n y construcci�n, y propiciando la elaboraci�n de relaciones operatorias y el enriquecimiento del vocabulario.
Como dice Lucio: � No se trata simplemente de la acci�n como recurso did�ctico, tal como se la concibe en las pedagog�as activas (mantener el ni�o activo para que no se distraiga), es algo m�s, es acci�n que le permite al sujeto establecer (construir) los nexos entre los objetos, y que, al interiorizarse y reflexionar, configura el conocimiento del sujeto�.
Durante el taller, se generaran estrategias y actividades de clase para el uso adecuado del geoplano, las regletas de cuisenaire, torres de hanoi y el tangram chino, para la ense�anza de las matem�ticas en la b�sica primaria y secundaria.

\begin{thebibliography}{1}
\end{thebibliography}
\end{document}