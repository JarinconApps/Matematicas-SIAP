\documentclass[12pt,twoside,twocolumn,english]{article}
\usepackage[T1]{fontenc}
\usepackage[latin9]{inputenc}
\usepackage{geometry}
\geometry{verbose,tmargin=3cm,bmargin=3cm,lmargin=2cm,rmargin=2cm}
\usepackage{fancyhdr}
\pagestyle{fancy}
\usepackage{babel}
\usepackage{url}
\usepackage{longtable}
\tolerance = 10000
\pretolerance = 10000
\setlength{\parindent}{0pt}
\usepackage[unicode=true,pdfusetitle,
 bookmarks=true,bookmarksnumbered=false,bookmarksopen=false,
 breaklinks=false,pdfborder={0 0 1},backref=false,colorlinks=false]
 {hyperref}

\makeatletter
%%%%%%%%%%%%%%%%%%%%%%%%%%%%%% User specified LaTeX commands.
\usepackage{fancyhdr}
\pagestyle{fancy}
\fancyhf{}

\fancyhead[RO]{5 Workshop de Educaci�n Matem�tica, Estad�stica y Matem�ticas - EMEM 2019, 5-7 de noviembre de 2019, Armenia, Colombia.}

\fancyfoot[LO]{  \rule[0.25ex]{1\columnwidth}{1pt} \\ Programa de Licenciatura en Matem�ticas, Universidad del Quind�o \hfill{} \thepage}

\setlength\columnsep{1cm}

\makeatother

\begin{document}
\title{Herramientas, recursos y teor�as para facilitar el proceso de ense�anza-aprendizaje de las matem�ticas por parte del grupo GEMAUQ}
\author{V�ctor Hugo Buitrago Caro\thanks{Universidad del Quind�o , vhbuitragoc@uqvirtual.edu.co}  }
\date{Noviembre del 2019}
\maketitle
\thispagestyle{fancy}
\subsection*{Palabras Clave}
Plan Padrino, Metacognici�n, Investigaci�n educativa, Investigaci�n en educaci�n, Educaci�n matem�tica, Herramientas, Did�ctica, Ense�anza-aprendizaje
\rule[0.25ex]{1\columnwidth}{1pt}
El grupo de Estudio e Investigaci�n en Educaci�n Matem�tica GEMAUQ, quiere compartir una peque�a muestra de las diferentes herramientas, recursos y teor�as que se manejan para facilitar el proceso de ense�anza-aprendizaje y el acompa�amiento que se desarrolla dentro del plan padrino con el fin de mejorar los niveles de desempe�o de los estudiantes en el �rea de matem�ticas y especialmente en los que presentan necesidades educativas especiales.
\begin{thebibliography}{1}
\bibitem{ref1}Brousseau, G. (2004). Research in mathematical education, Regular Lecture en el 10th. International Congress on Mathematics Education (ICME10). Dinamarca.
\bibitem{ref2}Duval, R. (2004). Semiosis y Pensamiento Humano Registros Semi�ticos y Apendizajes Intelectuales. Santiago de Cali: PeterLang S.A.
\bibitem{ref3}Godino, J et al. (2004). Did�ctica de las matem�ticas para docentes. Espa�a: Universidad de Granada.
\bibitem{ref4}Guilombo, M. (2011) La b�squeda de materiales para la ense�anza de la geometr�a con poblaci�n sorda de primer grado de educaci�n b�sica: un proceso de investigaci�n. Universidad Distrital Francisco Jos� de Caldas. Bogot� D.C.
\end{thebibliography}
\end{document}