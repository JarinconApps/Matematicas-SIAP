\documentclass[12pt,twoside,twocolumn,english]{article}
\usepackage[T1]{fontenc}
\usepackage[latin9]{inputenc}
\usepackage{geometry}
\geometry{verbose,tmargin=3cm,bmargin=3cm,lmargin=2cm,rmargin=2cm}
\usepackage{fancyhdr}
\pagestyle{fancy}
\usepackage{babel}
\usepackage{url}
\usepackage{longtable}
\tolerance = 10000
\pretolerance = 10000
\setlength{\parindent}{0pt}
\usepackage[unicode=true,pdfusetitle,
 bookmarks=true,bookmarksnumbered=false,bookmarksopen=false,
 breaklinks=false,pdfborder={0 0 1},backref=false,colorlinks=false]
 {hyperref}

\makeatletter
%%%%%%%%%%%%%%%%%%%%%%%%%%%%%% User specified LaTeX commands.
\usepackage{fancyhdr}
\pagestyle{fancy}
\fancyhf{}

\fancyhead[RO]{5 Workshop de Educaci�n Matem�tica, Estad�stica y Matem�ticas - EMEM 2019, 5-7 de noviembre de 2019, Armenia, Colombia.}

\fancyfoot[LO]{  \rule[0.25ex]{1\columnwidth}{1pt} \\ Programa de Licenciatura en Matem�ticas, Universidad del Quind�o \hfill{} \thepage}

\setlength\columnsep{1cm}

\makeatother

\begin{document}
\title{Ense�anza del sistema aditivo para docentes mediante el software SIMON MATH como ayuda did�ctica}
\author{Jhina Paola Ortiz Gallego \thanks{Universidad del quindio , jportizg@uqvirtual.edu.co}  , Kelly Valencia Villa \thanks{Universidad del quindio , kvvalenciav@uqvirtual.edu.co}   \\ Juli�n Andr�s Rinc�n Penagos\thanks{Universidad del Quind�o, jarincon@uniquindio.edu.co}  }
\date{Noviembre del 2019}
\maketitle
\thispagestyle{fancy}
\subsection*{Palabras Clave}
Ense�anza, estrategia didactica, ingenieria didactica, resolucion de problemas , secuencia didactica , sistema aditivo, software educativo, TIC
\rule[0.25ex]{1\columnwidth}{1pt}
La propuesta esta enmarcada en el desarrollo de un software (juego computarizado) para la ense�anza del sistema aditivo usando la teoria de las situaciones did�cticas de Brousseau. Esta necesidad nace de observar dificultades en el aprendizaje del sistema aditivo aunque esta problem�tica se a tratado de resolver a trav�s de diferentes juegos computarizados estas no han sido desarrolladas a partir del modelo de aprendizaje por adaptaci�n de las situaciones did�cticas, mediante lo cual se llega a la mecanizaci�n y repetici�n de las operaciones dadas en el esquema aditivo.
\begin{thebibliography}{1}
\bibitem{ref1}Anilema , J. (2016). Analisis, dise�o e implementaci�n de un software educativo para la ense�anza aprendizaje de la asignatura de matem�ticas dirigido a los estudiantes de tercer a�o de educaci�n b�sica.
\bibitem{ref2}Brousseau, G. (1999). Educaci�n y did�ctica de las matem�ticas.
\bibitem{ref3}Cardenas y Sarmiento . (2010). Elaboraci�n de un software educativo de matem�ticas para reforzar la ense�anza-aprendizaje mediante el juego interactivo, para ni�os de tercer a�o de educaci�n b�sica.
\bibitem{ref4}Cuicas, etal. (2007). El software Matem�tico como herramienta para el desarrollo de habilidades del pensamiento y mejoramiento del aprendizaje de las matem�ticas.
\bibitem{ref5}Vargas y Porras . (2015). �Resoluci�n de problemas en adici�n y sustracci�n de n�meros naturales mediante la aplicaci�n de componentes l�dicos, en estudiantes del grado sexto, del colegio comfatolima� .
\end{thebibliography}
\end{document}