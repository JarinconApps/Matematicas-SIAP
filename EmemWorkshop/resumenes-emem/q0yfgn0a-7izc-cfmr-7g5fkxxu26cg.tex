\documentclass[12pt,twoside,twocolumn,english]{article}
\usepackage[T1]{fontenc}
\usepackage[latin9]{inputenc}
\usepackage{geometry}
\geometry{verbose,tmargin=3cm,bmargin=3cm,lmargin=2cm,rmargin=2cm}
\usepackage{fancyhdr}
\pagestyle{fancy}
\usepackage{babel}
\usepackage{url}
\usepackage{longtable}
\tolerance = 10000
\pretolerance = 10000
\setlength{\parindent}{0pt}
\usepackage[unicode=true,pdfusetitle,
 bookmarks=true,bookmarksnumbered=false,bookmarksopen=false,
 breaklinks=false,pdfborder={0 0 1},backref=false,colorlinks=false]
 {hyperref}

\makeatletter
%%%%%%%%%%%%%%%%%%%%%%%%%%%%%% User specified LaTeX commands.
\usepackage{fancyhdr}
\pagestyle{fancy}
\fancyhf{}

\fancyhead[RO]{5 Workshop de Educaci�n Matem�tica, Estad�stica y Matem�ticas - EMEM 2019, 5-7 de noviembre de 2019, Armenia, Colombia.}

\fancyfoot[LO]{  \rule[0.25ex]{1\columnwidth}{1pt} \\ Programa de Licenciatura en Matem�ticas, Universidad del Quind�o \hfill{} \thepage}

\setlength\columnsep{1cm}

\makeatother

\begin{document}
\title{Planes de clase y uso del tablero acorde a la did�ctica de las matem�ticas en el Jap�n para la b�sica primaria}
\author{Reina Yamashita\thanks{Voluntaria de JICA, reinayamashita33@gmail.com}  , Hiro Iwamoto\thanks{Voluntario de JICA, hirokiiwamoto5@gmail.com}   \\ Magda Lorena L�pez Osorio\thanks{Colegio Antonio Nari�o, Calarca, magdalo_@hotmail.com}  , Martha Cecilia Ram�rez Rodriguez\thanks{Rom�n Mar�a Valencia de Calarc�, marthace87@gmail.com}  }
\date{Noviembre del 2019}
\maketitle
\thispagestyle{fancy}
\subsection*{Palabras Clave}
Planes de Clase, Uso del tablero, Did�ctica de las Matem�ticas
\rule[0.25ex]{1\columnwidth}{1pt}
Planear las clases es el eje central en toda pr�ctica docente, porque de ella depende el �xito de su labor. Una �ptima planeaci�n, con dise�o  de estrategias que despiertan motivaci�n y potencialicen competencias cognitivas y actitudinales, conlleva a un quehacer organizado, coherente y cient�fico en el aula.
De otra parte, el tablero por ser la herramienta did�ctica disponible en todas las aulas y la m�s utilizada por los docentes y estudiantes, amerita que se identifiquen y desarrollen t�cnicas que permitan descubrir su riqueza y verdadero valor pedag�gico.
En este taller se har� una peque�a rese�a de la metodolog�a de la ense�anza de las matem�ticas en el Jap�n, seguida de clases demostrativas sobre el plan de la clase y el uso del tablero, finalizando cada jornada con talleres pr�cticos de planeaci�n y ejecuci�n de algunas clases de grados de los 1� a 5� de la b�sica primaria.

\begin{thebibliography}{1}
\end{thebibliography}
\end{document}